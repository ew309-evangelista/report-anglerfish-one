The system must autonomously scan for, identify, and track discrete targets. Once a target is identified, the turret must center itself on target and, if the turret has a reasonable chance to hit the target, fire. The system must be able to distinguish between valid targets and those that should not be fired upon, and operate accordingly..


https://www.instructables.com/id/Autonomous-Sentry-Turret/
This turret design uses a laser range finder and infrared motion sensor to determine targets. The turret sweeps a range of 100 degrees with 10 degree intervals, storing range data. Once data is recorded for the entire sweep and stored data, the turret starts sweeping the same range, and if the range is different from the stored value, the turret fires a third of its clip down the bearing. The turret then waits for the moment sensor to be tripped to begin sweeping again.
	The greatest issue this turret has with our problem statement is a lack of discrimination. A difference in rangefinder data will trigger the turret, for example, a door that was either opened or closed mid sweep would be treated as a target. Also, if a person were to walk by the turret during its initial sweep, when the ranges are being determined, the turret would, on its second sweep register a target and begin firing. Furthermore, once the turret starts to fire, the bearing is static, so many shots would be wasted on a moving target. If instead a change in range was coupled with the IR sensor, the system could avoid some false positives.

https://www.instructables.com/id/Face-Tracking-Pan-Tilt-Camera/
This is a gimbal system for a camera meant to track faces (or other objects depending on training). The system must be provided with several photos of whatever image it is meant to detect. The system uses machine learning to determine what constitutes a face (or target of choice), then gives the coordinates of the target relative to the camera’s center. This data is then used to center the servo, using a proportional control.
	The low point of this solution seems to be the control algorithm. The demonstration videos show slow and imperfect response (with a large steady state error). Furthermore, the system in this project is only built to be able to detect one type of target at a time. It can only positively identify one type of target, and cannot determine between different types of targets. Furthermore, having more than one target in frame could cause the system to oscillate between the two.

https://www.instructables.com/id/Building-a-moving-and-tracking-Portal-Turret/
This is a 3D printed turret system that can be operated in two modes autonomously or by joystick controlled. When connected to a computer it uses software to detect and track a target designated target. It does this by constantly taking a stream of data and converting it to motor movement. The turret is smart enough to recognize when the target is moving and when the target moves out of the view of the camera.  The software that runs this turret is highly similar to the one that we will use in our own project except it is written in Arduino. 
	The biggest weakness of the design however is that it requires the use of a 3D printer to make the pieces with enough precision that the guns, motors, and  electrical work can all fit within the shell. Another issue with this design for our project is it would require us to learn a new language to code the turret.  
https://www.instructables.com/id/Robotic-Talking-Turret/
This turret design uses a rangefinder along with the Stampy method to track targets. The rangefinder is constantly sweeping back and forth until it finds a target. Once the target is found the turret starts sweeping and firing. The design of the turret also allows it to know when it is being picked up or tipped over. Both of these actions would stop the rangefinder from moving. 
The largest issues in this design is that it will track and fire at an object whether it be a target or not as long as it is within the set range of the range finder. Meaning that if someone where to walk past or another object be within the set range the turret would track it and fire. Another large issue that this design has is that it fires as soon as an object is within range no matter where the turret is a lined.





The system uses a pico usb webcam mounted above on a rail above the barrel.

Camera mount

To get an image into matlab, we initialized a usb webcam, then pulled the color data from a frame taken by the camera. Every loop of the program, we pull a new image, display its color data, and pass that color data on to our thresholding function.
Once an image is acquired by matlab, its color data is sent to our thresholding function. From here, a mask is created, marking all pixels with the right color data (orange tones) as ones, and all other pixels black. From here, we fill any small holes that may exist inside of larger swathes of the correct color. Then, all connected regions above a constant minimum area are labelled, and their properties recorded (centroid,area, eccentricity). 
The image processing procedure checks the size and shape of all areas with the proper color profiles. If potential targets are outside the bounds we have placed on area, or are too irregularly shaped (the desired targets are circles), the script drops them from the list of desired targets. For now we simply target the first viable target on the list. We are considering switching to focusing on the target closest to the center of the frame, as we are unsure if our current method will lead to oscillation between multiple targets.
To ensure the target acquisition works for multiple targets, we tried using a couple different orange circles to make sure they all got picked up by the acquisition software. 
maxArea, minArea, and maxEcc constants need to be fine tuned in the future. Once a rough range of the target is known, we can adjust the area constants to reflect this range. For now we have filler values of minArea=50, maxArea=7500, and maxEcc=0.75.

Midn Malatesta and Hendricks both developed their own acquisition scripts. Midn Malatesta took charge of the color/image processing. Midn Hendricks took charge of the range calibration. 





The turret needs to run two motors in the gun to properly fire. The mbed can put out small amounts of voltage and very low amounts of current. The motor requires much higher current and voltage than the mbed can supply. So, a firing circuit is used to control the  motors from the mbed, and provide the motors with the proper voltage and current.

Figure 1: Firing Circuit Schematic

Figure 2: Firing Circuit

The pulldown resistor is to put a high resistance between the mbed pin and ground. This means that the transistor gate pin is not connected directly to ground, so the transistor will close when the mbed pin is high.
The mbed is incapable of directly driving the motors, but is capable of controlling a mosfet on/off. So, the mosfet functions as a switch, controlled by the mbed, for the motor circuits, allowing the mbed to control otherwise oversized loads.
A motor is a high inductance load. When the mosfet is off, the current produced by the inductive components of the motor needs some way to dissipate. The flyback diode and snubber allow for this induced current to disperse. 
Statement of Contribution
MIDN Malatesta created the first and third (backup) mosfet circuits. Midn Hendricks created the second by copying the first circuit. The overall contribution was roughly 60% Malatesta, 40% Hendricks.





To determine the motor requirements, we first defined the desired response of the motor. Then, we computed a rough estimate of the moment of inertia using the mass and geometry of various parts of the turret system. This allowed us to determine what torque would be necessary to drive the motor according to our specifications. (put spreadsheet in appendix on overleaf).
	We chose a motor that could fulfil all the desired requirements. Both motors on the decision tree met requirements, so to reduce costs, we chose the motor that exceeded the requisites by less. Our final choice of motor, the ______ is the one already employed in the turret, our decision simply confirmed this was the right choice. (Cannot access motor datasheets right now- had bookmarked the page but it’s on intranet).

Motor torque, speed, and current information was all available from the data sheet on the robotics and controls website. (Currently unavailable, will create plots once get data).

The turret motor is controlled by the mbed and a TD340 motor driver. The motor requires higher voltage and current than the mbed can supply, leading to the intermediary. The mbed supplies two signals, a digital out for direction, and a pwm signal for speed. We chose to use pwm from the mbed instead of analog out because the analog signal requires two conversions, increasing chances for error. The analog and digital speed control use two different input pins on the motor driver. A pulldown resistor had to be placed on the speed control pin to prevent the system actuating on startup. The connection to ground reduced the effect of noise on the system.
D: To change the turret direction, the mbed digital out is changed from high to low. As this leads to the current through the motor being reversed, the turret behaves the same in both directions.


