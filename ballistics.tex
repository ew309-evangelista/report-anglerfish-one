\subsection{Ballistics}
% Need words here that explain why we care about ballistics, bias and precision error, and circular error probable. 
To determine whether or not the system would actually be able to hit a target it aimed at, the bullet ballistics were processed. From this data, we are able to calculate a circular error probable, and then calculate the number of shots needed to give a certain probability of hitting a target. The ballistics analysis also gives us our x and y bias, which allows us to correct for the path the bullet takes.

The full ballistics test procedure is in appendix~\ref{app:A}. The test consisted of filing multiple shots from a certain range, recording and analyzing the results and creating trends for bullet spread and bias due to distance.

Due to coronavirus, ballistics data were provided by the instructors. .
\begin{figure}[h!]
    \centering
    \includegraphics[scale=.4]{raw impact.jpg}
    \caption{Raw Ballistic Data }
    \label{fig:raw datal}
\end{figure}
\begin{figure}[h!]
    \centering
    \includegraphics[scale=.4]{balistics trends.jpg}
    \caption{Ballistics Trends}
    \label{fig:my_label}
\end{figure}
\begin{figure}[h!]
    \centering
    \includegraphics[scale=.4]{cep's.jpg}
    \caption{Circular Error Probability}
    \label{fig:ceps}
\end{figure}